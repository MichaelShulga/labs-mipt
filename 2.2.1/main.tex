\documentclass[a4paper,12pt]{article} % тип документа

% Поля страниц
\usepackage[left=2.5cm,right=2.5cm,top=1.5cm,bottom=2cm,bindingoffset=0cm]{geometry}
    
% Отступ после заголовка
\usepackage{indentfirst}

% Картинки
\usepackage{graphicx}
\graphicspath{{images/}}
\usepackage{placeins}

% Таблицы
\usepackage{booktabs}
% \usepackage{floatrow}
\usepackage{subcaption}

% Русский язык
\usepackage{cmap}  % поиск в PDF
\usepackage{mathtext}  % русские буквы в формулах
\usepackage[T2A]{fontenc}  % кодировка
\usepackage[utf8]{inputenc}  % кодировка исходного текста
\usepackage[english,russian]{babel}  % локализация и переносы

% Математика
\usepackage{amsmath}

% Ссылки TODO
% \usepackage[unicode=true]{hyperref}
% \usepackage[T1]{fontenc}

\begin{document}

\begin{center}   
	\large{Лабораторная работа № 2.2.1\\\textbf{Исследование взаимной диффузии газов}}\\
\end{center}

\section{Аннотация}

\noindent\textbf{Цель работы:}
1) регистрация зависимости концентрации гелия в воздухе от времени с помощью датчиков теплопроводности при разных начальных давлениях смеси газов; 2) определение коэффициента диффузии по результатам измерений.
	
\smallskip
\noindent\textbf{В работе используются:}
измерительная установка; форвакуумный насос; баллон с газом (гелий); манометр; источник питания; магазин сопротивлений; гальванометр; секундомер.

\section{Теоретические сведения}

\textit{Диффузией}  называют самопроизвольное взаимное проникновение веществ друг в друга происходящее вследствие хаотичного теплового движения молекул. При перемешивании молекул разного сорта говорят о взаимной (или концентрационной) диффузии. В системе, состоящей из двух компонентов a и b (бинарная смесь), плотности потоков частиц в результате взаимной диффузии определяются законом Фика:
\begin{equation}
    j_a = -D \frac{\partial n_a}{\partial x}, \, j_b = -D \frac{\partial n_b}{\partial x},
\end{equation}
где $D$ — \textit{коэффициент взаимной диффузии компонентов}. Знак <<минус>> отражает тот факт, что диффузия идёт в направлении выравнивания концентраций. Равновесие достигается при равномерном распределении вещества по объёму.

В данной работе исследуется взаимная диффузия гелия и воздуха. Отметим, что давление и температура в системе предполагаются неизменным: $P_0 = (n_{He}+n_{в})kT = const$, где $n_{He}$  и $n_{в}$ -- концентрации диффундирующих газов. Поэтому для любых изменений концентраций справедливо $\Delta n_{в} = -\Delta n_{He}$. Следовательно, достаточно ограничиться описанием диффузии одного из компонентов, например гелия.

Приведём теоретическую оценку для коэффициента диффузии. В работе концентрация гелия, как правило, мала ($n_{He} \ll n_{в}$). Кроме того, атомы гелия легче молекул, составляющих воздух ($m_{He} \ll m_{N_2}, m_{O_2}$), значит их средняя тепловая скорость велика по сравнению с остальными частицами. Поэтому перемешивание газов в работе можно приближенно описывать как диффузию примеси лёгких частиц He на практически стационарном фоне воздуха. Коэффициент диффузии в таком приближении равен
\begin{equation}\label{D}
    D = \frac{1}{3} \lambda \langle v \rangle,
\end{equation}
где $\lambda = \frac{1}{n\sigma}$ -- длина свободного пробега диффундирующих частиц; $\langle v \rangle = \sqrt{\frac{8kT}{\pi m}}$ -- их средняя тепловая скорость.

Предпологая, что процесс диффузии будет квазиостационарным, можно показать, что разность концентраций будет убывать по экспоненциальному закону
\begin{equation}\label{Delta_n}
    \Delta n = \Delta n_0 e^{-t / \tau},
\end{equation}
где $\tau$ -- характерное время выравнивания концентраций между сосудами, определяемое следующей формулой
\begin{equation}\label{Tau}
    \tau = \frac{1}{D} \frac{VL}{2S}.
\end{equation}

\section{Используемое оборудование}

\begin{figure}[h]
    \center{\includegraphics[width=\textwidth]{установка}}
    \caption{Установка}
    \label{установка}
\end{figure}

Здесь $V_1,\; V_2$ -- два сосуда с примерно равным объемом, в которые мы будем загонять воздух и гелий.

Данная конструкция позволяет провести диффузию, которая возможна только при равенстве давлений.

Основное оборудование, с помощью которого мы будем снимать измерения -- датчики теплопроводности, через которые пропускают ток. Они подключены к мосту, который позволяет нам устанавливать начальное равновесное состояние.

При изменении концентрации в колбах вольтметр покажет нам разность напряжений на датчиках, что, из-за их конструкции, означает разность концентраций. 

С помощью изменения напряжения мы и будем изучать процесс диффузии, т.к. во время ее протекания концентрации газов начинают устанавливаться, что заметно на графике разницы напряжений от времени.

% \section{Методика измерений}

\newpage

\section{Результаты измерений и обработка данных}

\subsection*{Параметры установки}

\begin{equation}\label{V}
    V_1 \approx V_2=1200.0 \pm 3.0 \ см^3

\end{equation}

\begin{equation}\label{LdivS}
    L/S=5.5 \pm 0.5 \ см^{-1}

\end{equation}

\begin{equation}\label{T}
    T=295.0 \pm 1.0 \ K

\end{equation}

\begin{equation}\label{P}
    P=745.70 \pm 0.10 \ торр

\end{equation}

\subsection*{Зависимость показаний вольтметра от времени}

\begin{figure}[h!]
    \centering
    \includegraphics[width=0.85\textwidth]{lnU(t).pdf}
    \caption{Логарифмическая зависимость}\label{lnUt}
\end{figure}

Используя формулы (\ref{Delta_n}) и (\ref{Tau}), по угловым коэффициентам и известным геометрическим параметрам установки (\ref{V}) и (\ref{LdivS}) рассчитаем коэффициенты взаимной диффузии при выбранных рабочих давлениях:

\begin{table}[h!]
    \centering
    \begin{tabular}{ll}
\toprule
$D$, $см^2 \cdot с^{-1}$ & $P$, $торр$ \\
\midrule
$9.3 \pm 0.8$ & $40 \pm 4$ \\
$4.7 \pm 0.4$ & $80 \pm 4$ \\
$3.38 \pm 0.31$ & $120 \pm 4$ \\
$2.39 \pm 0.22$ & $160 \pm 4$ \\
\bottomrule
\end{tabular}

    \caption{Коэффициенты взаимной диффузии при выбранных рабочих давлениях}
\end{table}

\FloatBarrier

\subsection*{Коэффициент диффузии}

\begin{figure}[h!]
    \centering
    \includegraphics[width=0.85\textwidth]{D(1divP).pdf}
    \caption{Зависимость коэффициента диффузии от обратного давления}
\end{figure}

Экстраполируя график к атмосферному давлению (\ref{P}):
\begin{equation}\label{D0}
    D_{He-в}^{атм}=0.65 \pm 0.16 \ см^2 \cdot с^{-1}

\end{equation}

\FloatBarrier

\subsection*{Длина свободного пробега и эффективное сечение}

Оценим длину свободного пробега атомов гелия в воздухе по формуле (\ref{D}):
\begin{equation}\label{lambda0}
    \lambda_{He}^{атм}=\left(1.6 \pm 0.4\right) \times 10^{2} \ нм

\end{equation}

А также эффективное сечение столкновений атомов гелия с молекулами воздуха:
\begin{equation}\label{sigma0}
    \sigma_{He-в}^{атм}=0.26 \pm 0.07 \ нм^2

\end{equation}

\section{Обсуждение результатов}

Из графика (рис. \ref{lnUt}) видно, что процесс диффузии подчиняется закону (\ref{Delta_n}).

Вычислен коэффициент диффузии при атмосферном двалении (\ref{D0}), значение которого хорошо согласуется с табличным:
\begin{equation}\label{D_real}
    D_{He-в}^{табл}=0.620 \pm 0.010 \ см^2 \cdot с^{-1}

\end{equation}

Вычислены длина свободного пробега атомов гелия в воздухе (\ref{lambda0}), а также эффективное сечение столкновений атомов гелия с молекулами воздуха (\ref{sigma0}). Значение последнего хорошо согласуется с теоретически вычисленным по порядку:
\begin{equation}\label{sigma_real}
    \sigma_{He-в}^{теор}=\pi d_{He}^2=0.1385 \pm 0.0026 \ нм^2

\end{equation}

\end{document}