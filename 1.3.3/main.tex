\documentclass[a4paper,12pt]{article} % тип документа

% Поля страниц
\usepackage[left=2.5cm,right=2.5cm,
    top=2cm,bottom=2cm,bindingoffset=0cm]{geometry}
    
%Пакет дял таблиц   
\usepackage{multirow} 
    
%Отступ после заголовка    
\usepackage{indentfirst}


% Рисунки
\usepackage{floatrow,graphicx,calc}
\usepackage{wrapfig}

%%% Работа с картинками
\usepackage{graphicx}  % Для вставки рисунков
\graphicspath{{images/}{images2/}}  % папки с картинками
\setlength\fboxsep{3pt} % Отступ рамки \fbox{} от рисунка
\setlength\fboxrule{1pt} % Толщина линий рамки \fbox{}
\usepackage{wrapfig} % Обтекание рисунков и таблиц текстом

% Создаёем новый разделитель
\DeclareFloatSeparators{mysep}{\hspace{1cm}}

% Ссылки?
\usepackage{hyperref}
\usepackage[rgb]{xcolor}
\hypersetup{				% Гиперссылки
    colorlinks=true,       	% false: ссылки в рамках
	urlcolor=blue          % на URL
}


%  Русский язык
\usepackage[T2A]{fontenc}			% кодировка
\usepackage[utf8]{inputenc}			% кодировка исходного текста
\usepackage[english,russian]{babel}	% локализация и переносы


% Математика
\usepackage{amsmath,amsfonts,amssymb,amsthm,mathtools}

%%% Дополнительная работа с математикой
\usepackage{amsmath,amsfonts,amssymb,amsthm,mathtools} % AMS
\usepackage{icomma} % "Умная" запятая: $0,2$ --- число, $0, 2$ --- перечисление


% Что-то 
\usepackage{wasysym}

% Мое
% \usepackage[pipeTables=true]{markdown}
\usepackage{longtable}
\usepackage{booktabs}


\begin{document}
% \begin{center}
% 	\footnotesize{ФЕДЕРАЛЬНОЕ ГОСУДАРСТВЕННОЕ АВТОНОМНОЕ ОБРАЗОВАТЕЛЬНОЕ 			УЧРЕЖДЕНИЕ ВЫСШЕГО ОБРАЗОВАНИЯ}\\
% 	\footnotesize{МОСКОВСКИЙ ФИЗИКО-ТЕХНИЧЕСКИЙ ИНСТИТУТ\\(НАЦИОНАЛЬНЫЙ 			ИССЛЕДОВАТЕЛЬСКИЙ УНИВЕРСИТЕТ)}\\
% 	\footnotesize{ФАКУЛЬТЕТ ОБЩЕЙ И ПРИКЛАДНОЙ ФИЗИКИ\\}
% 	\hfill \break
% 	\hfill \break
% 	\hfill \break
% 	\hfill \break
% \end{center}


% \begin{figure*}[h]
%     \centering
%     \includegraphics*[width=10cm,height=7cm,keepaspectratio]{mipt_eng_text_png.png}
%     \label{fig:my_label}
% \end{figure*}


\begin{center}   
    \hfill \break
	\hfill \break
	\hfill \break
	\hfill \break
	\large{Лабораторная работа № 1.3.3\\\textbf{Измерение вязкости воздуха по течению в тонких трубках}}\\
	% \hfill \break
	% \hfill \break
	% \hfill \break
	% \hfill \break
	% \begin{flushright}
	% 	Баранов Даниил\\
	% 	Группа Б02-103
	% \end{flushright}
	% \hfill \break
	% \hfill \break
	% \hfill \break
\end{center}
% \hfill \break
% \hfill \break
% \hfill \break
% \hfill \break
% \begin{center}
% 	Долгопрудный, 2022 г.
% \end{center}
% \thispagestyle{empty}



% \newpage
\textbf{Цель работы:} экспериментально исследовать свойства течения газов по тонким трубкам при различных числах Рейнольдса; выявить область применимости закона Пуазейля и с его помощью определить коэффициент вязкости воздуха.\hfill
\break
	
\textbf{В работе используются:} система подачи воздуха (компрессор, поводящие трубки); газовый счетчик барабанного типа; спиртовой микроманометр с регулируемым наклоном; набор трубок различного диаметра с выходами для подсоединения микроманометра; секундомер.


\section{Теоретическая часть}
Рассмотрим движение вязкой жидкости или газа по трубке круглого сечения. При малых скоростях потока движение оказывается ламинарным (слоистым), скорости частиц меняются по радиусу и направлены вдоль оси трубки. С увеличением скорости потока движение становится турбулентным, а слои перемешиваются. При турбулентном движении скорость в каждой точке быстро меняет величину и направление, сохраняется только средняя величина скорости.

Характер движения газа (или жидкости) в трубке определяется безразмерным числом Рейнольдса:
\[
	Re = \frac{vr\rho}{\eta}
\]
где $v$ -- скорость потока, $r$ -- радиус трубки, $\rho$ -- плотность движущейся среды, $\eta$ -- её вязкость. В гладких трубах круглого сечения переход от ламининарного движения к турбулентному происходит при $Re \approx 1000$.

При ламинарном течении объем газа $V$, протекающий за время $t$ по трубе длиной $l$, определяется формулой Пуазейля:
\begin{equation}
	Q = \frac{\pi r^4}{8 \Delta l \eta}(P_1 - P_2)
\end{equation}
В этой формуле $P_1 - P_2$ -- разность давлений в двух выбранных сечениях 1 и 2, расстояние между которыми равно $\Delta l$. Величину $Q$ обычно называют расходом. Формула (1) позволяет определять вязкость газа по его расходу.

Отметим условия, при которых справедлива формула (1). Прежде всего необходимо, чтобы с достаточным запасом выполнялось неравенство $Re < 1000$. Необходимо также, чтобы при течении не происходило существенного изменения удельного объёма газа (при выводе формулы удельный объём считался постоянным). Для жидкости это предположение выполняется практически всегда, а для газа --- лишь в тех случаях, когда перепад давлений вдоль трубки мал по сравнению с самим давлением. В нашем случае давление газа равно атмосферному ($10^3$ см вод. ст.), а перепад давлений составляет не более 10 см вод. ст., т. е. менее 1\% от атмосферного. Формула (1) выводится для участков трубки, на которых закон распределения скоростей газа по сечению не меняется при двидении вдоль потока.
% \begin{figure}[H]
% \center
% \includegraphics[scale=1]{potok.png}
% \caption{Формирование потока газа в трубке круглого сечения}
% \end{figure}
При втекании газа в трубку из большого резервуара скорости слоёв вначале постоянны по всему направлению. По мере продвижения газа по трубке картина распределения скоростей меняется, так как сила трения о стенку тормозит прилежащие к ней оси. Характерное для ламинарного течения параболическое распределение скоростей устанавливается на некотором расстоянии $a$ от входа в трубку, которое зависит от радиуса трубки $r$ и числа Рейнольдса по формуле 
\begin{equation}
	a \approx 0.2rRe
\end{equation}
Градиент давления на участке формирования потока оказывается больше, чем на участке с установившимся ламинарным течением, что позволяет разделить эти участки экспериментально. Формула (2) даёт возможность оценить дину участка формирования.

\section{Экспериментальная установка}

Схема экспериментальной установки изображена на Рис. \ref{233}. Поток воздуха
под давлением, немного превышающим атмосферное, поступает через газовый счётчик в тонкие металлические трубки. Воздух нагнетается компрессором, интенсивность его подачи регулируется краном К. Трубки снабжены
съёмными заглушками на концах и рядом миллиметровых отверстий, к которым можно подключать микроманометр. В рабочем состоянии открыта заглушка на одной (рабочей) трубке, микроманометр подключён к двум её выводам, а все остальные отверстия плотно закрыты пробками.

\begin{figure}[H]
    \centering
    \includegraphics[scale=0.65]{expust.PNG}
    \caption{Экспериментальная установка}
    \label{233}
\end{figure}

Перед входом в газовый счётчик установлен водяной U-образный манометр. Он служит для измерения давления газа на входе, а также предохраняет
счётчик от выхода из строя. При превышении максимального избыточного
давления на входе счётчика ($\sim$ 30 см вод. ст.) вода выплёскивается из трубки
в защитный баллон Б, создавая шум и привлекая к себе внимание экспериментатора.


\textbf{Газовый счётчик.} В работе используется газовый счётчик барабанного
типа, позволяющий измерять объём газа $\Delta V$ прошедшего через систему. Измеряя время $\Delta t$ при помощи секундомера, можно вычислить средний объёмный расход газа $Q = \Delta V/ \Delta t$ (для получения массового расхода [кг/с] результат
необходимо домножить на плотность газа $\rho$).

\begin{figure}[H]
    \centering
    \includegraphics[scale=0.65]{gascounter.PNG}
    \caption{Газовый счетчик}
    \label{333}
\end{figure}


Работа счётчика основана на принципе вытеснения: на цилиндрической ёмкости жёстко
укреплены лёгкие чаши (см. Рис. \ref{333}, где для
упрощения изображены только две чаши), в которые поочередно поступает воздух из входной
трубки расходомера. Когда чаша наполняется,
она всплывает и её место занимает следующая
и т.д. Вращение оси предаётся на счётно-суммирующее устройство.
Для корректной работы счётчика он должен
быть заполнен водой и установлен горизонтально по уровню (подробнее см. техническое
описание установки).

\textbf{Микроманометр.} В работе используется жидкостный манометр с наклонной трубкой. Разность давлений на входах манометра измеряется по высоте
подъёма этилового спирта. Регулировка
наклона позволяет измерять давление в различных диапазонах.

На крышке прибора установлен трехходовой кран, имеющий два рабочих
положения — (0) и (+). В положении (0) производится установка мениска жидкости на ноль, что необходимо сделать перед началом работы (в процессе работы также рекомендуется периодически проверять положение нуля). В положении (+) производятся измерения.
\newpage


\section{Результаты измерений и обработка данных}

\subsection*{Параметры окружающей среды}

$T=(299.25 \pm 0.10)$ К

$P=(97610 \pm 10)$ Па

\subsection*{Ламинарное и турбулентное течение}

\begin{table}[ht!]
\caption{$d=(3.95 \pm 0.05)$ мм, $l=(50.0 \pm 1.0)$ см}
\begin{tabular}{ cc }
\begin{tabular}{c|c|c|c}
\toprule
$\Delta t$, c & $N$ & $Q$, мл/с & $\Delta P$, Па \\
\midrule
12.80 & 63 & 78.12 & 122.45 \\
17.30 & 47 & 57.80 & 91.35 \\
19.90 & 40 & 50.25 & 77.75 \\
24.20 & 34 & 41.32 & 66.09 \\
26.60 & 30 & 37.59 & 58.31 \\
33.70 & 24 & 29.67 & 46.65 \\
41.70 & 20 & 23.98 & 38.87 \\
52.80 & 15 & 18.94 & 29.16 \\
84.20 & 10 & 11.88 & 19.44 \\
\bottomrule
\end{tabular}

\begin{tabular}{rrrr}
\toprule
$\Delta t$, $c$ & $N$ & $\Delta P$, $Па$ & $Q$, $мл/с$ \\
\midrule
9.6 & 101 & 196.3 & 104.2 \\
9.1 & 120 & 233.2 & 109.9 \\
8.5 & 154 & 299.3 & 117.6 \\
8.0 & 171 & 332.4 & 125.0 \\
7.8 & 186 & 361.5 & 128.2 \\
7.4 & 206 & 400.4 & 135.1 \\
6.7 & 247 & 480.1 & 149.3 \\
6.2 & 277 & 538.4 & 161.3 \\
\bottomrule
\end{tabular}

\end{tabular}
\end{table}

\begin{table}[ht!]
\caption{$d=(3.95 \pm 0.05)$ мм, $l=(90.0 \pm 1.0)$ см}
\begin{tabular}{ cc }
\begin{tabular}{c|c|c|c}
\toprule
$\Delta t$, c & $N$ & $Q$, мл/с & $\Delta P$, Па \\
\midrule
16.60 & 93 & 60.24 & 180.76 \\
19.20 & 80 & 52.08 & 155.50 \\
22.00 & 70 & 45.45 & 136.06 \\
25.30 & 60 & 39.53 & 116.62 \\
30.80 & 50 & 32.47 & 97.18 \\
42.90 & 35 & 23.31 & 68.03 \\
78.60 & 20 & 12.72 & 38.87 \\
\bottomrule
\end{tabular}

\begin{tabular}{c|c|c|c}
\toprule
$\Delta t$, c & $N$ & $Q$, мл/с & $\Delta P$, Па \\
\midrule
12.10 & 130 & 82.64 & 252.68 \\
11.10 & 143 & 90.09 & 277.95 \\
10.50 & 155 & 95.24 & 301.27 \\
10.20 & 165 & 98.04 & 320.71 \\
9.90 & 176 & 101.01 & 342.09 \\
9.70 & 185 & 103.09 & 359.58 \\
9.40 & 212 & 106.38 & 412.06 \\
\bottomrule
\end{tabular}

\end{tabular}
\end{table}

\begin{table}[ht!]
\caption{$d=(5.30 \pm 0.05)$ мм, $l=(40.0 \pm 1.0)$ см}
\begin{tabular}{ cc }
\begin{tabular}{rrrr}
\toprule
$\Delta t$, $c$ & $N$ & $\Delta P$, $Па$ & $Q$, $мл/с$ \\
\midrule
7.0 & 35 & 68.0 & 142.9 \\
7.5 & 30 & 58.3 & 133.3 \\
8.8 & 25 & 48.6 & 113.0 \\
10.9 & 20 & 38.9 & 91.7 \\
13.9 & 15 & 29.2 & 71.9 \\
\bottomrule
\end{tabular}

\begin{tabular}{rrrr}
\toprule
$\Delta t$, $c$ & $N$ & $\Delta P$, $Па$ & $Q$, $мл/с$ \\
\midrule
6.3 & 53 & 103.0 & 158.7 \\
5.4 & 75 & 145.8 & 185.2 \\
4.6 & 102 & 198.3 & 217.4 \\
3.8 & 142 & 276.0 & 263.2 \\
3.6 & 166 & 322.7 & 277.8 \\
\bottomrule
\end{tabular}

\end{tabular}
\end{table}

\end{document}